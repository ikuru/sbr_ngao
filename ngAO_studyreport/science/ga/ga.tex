% 20131222
\def\thisdir{science/ga/}
%\documentclass[12pt,letter,times]{cv}
%\usepackage[T1]{fontenc}
%\usepackage{arev}
%\usepackage{setspace}
%\usepackage{graphicx}
%\usepackage{natbib}
%\usepackage{amssymb}
%\usepackage{float}
%\singlespacing
\newcommand*{\bull}{{\scriptsize $\bullet$\hspace{0.15cm}}}

\section{The Galaxy and nearby galaxies
\label{sec:ga}}

\noindent
%\\vspace{5mm}
%\large
\begin{center}
%% Authors
{\bf Alan Walker McConnachie\footnote{NRC Herzberg Institute of
 Astrophysics, Victoria, BC, Canada
 \noindent(alan.mcconnachie@nrc-cnrc.gc.ca)}, 
Masashi Chiba\footnote{Tohoku University, Sendai 980-8578, Japan}, 
Shogo Nishiyama\footnote{NAOJ, Tokyo, Japan}, 
Kim Venn\footnote{University of Victoria, BC, Canada}}
\end{center}
\vspace{0.5cm}

%\author{Alan Walker McConnachie}

%\begin{document}

%\maketitle

%\vspace{-0.25cm}

This report summarizes some relevant science considerations for the
development of a new near infrared (NIR) instrument capability on the
Subaru telescope. The first section discusses possible science drivers
for resolved stellar population studies of our Galaxy, the nearest
galaxies, and related science. In many respects, these science themes
are the ideal science drivers for a NIR instrument assisted by Ground
Layer Adaptive Optics (GLAO) since the science is fundamentally
wide-field and confusion limited. Thus GLAO serves not just as a way of
increasing sensitivity for fainter sources, but is also crucial for
correctly identifying brighter sources. The second section of this
document presents a more general summary of the overall science context
that this new instrument will operate within, and makes a few general
points for future consideration. 

\subsection{The Galaxy and nearby galaxies: Precision insights into galaxy evolution in the nearby Universe}

The nearest galaxies - including our own Galaxy - provide the most
detailed view available on the formation, structure and evolution of
galaxies spanning a broad range of mass and morphological types. This is
primarily achieved by obtaining precision data on the positions, colors,
compositions and kinematics of the many thousands and millions of
individual resolved stars that are accessible to us through observations
on 4 - 10m class facilities. In the future, TMT and the EELT will push
the volume in which this science is possible out to the Virgo cluster
and beyond, bringing within reach nearly every type of major environment
in which galaxies are found, and increasing by orders of magnitude the
number of galaxies for which we can resolve their individual stellar
constituents. Subaru must aim to play a key role in this transformative
and growing research area. TMT and EELT are able to observe ultra-deep
in relatively narrow fields; the focus of Subaru science, therefore,
should be to {\it complement these facilities by providing a wide field
perspective of the stellar populations of the nearest galaxies.} 

\subsubsection{The Galactic Center}

The Galactic Center is a primary target for NIR capabilities on large
telescopes. Massive extinction means that optical surveys are severely
hampered. Depending on the specific science of interest, the potential
area that requires to be surveyed is giant, and demands wide-field
capabilities. This is true not just for imaging but also for
spectroscopy. For many resolved stellar population goals, spectroscopy
is better done in the optical; as well as coinciding with the peak of
the spectral energy distribution (SED), the absence of many strong sky
lines and the high number of atomic transitions  makes the optical
extremely useful, in the absence of other contributing factors. However,
extinction is sufficiently strong in the Galactic Center that the {\it
only} way to obtain spectra for many of the stars is to go to NIR
wavelengths (the recent success of APOGEE, for example, testifies to
this). 

\begin{figure*}[htb]
  \begin{center}
%    \includegraphics[angle=0, width=8cm]{GC_NB}\vspace{-1cm}
    \caption{Identification of intermediate-age stellar populations in
   the Galactic center through narrow band imaging at NIR
   wavelengths. See Nishiyama \& Schodel 2013 for more details.}
  \end{center}
\end{figure*}

There are many precision studies of the Galactic center planned with
forthcoming facilities, for example TMT, in order to look at the proper
motions of stars orbiting the super-massive black hole. However, in
terms of {\it wide-field} science, there are important questions
relating to the bulge stellar populations, star clusters near the center
of the Galaxy (see also next Section) and the connection of the nuclear
star cluster (NSC) to the Galactic SMBH and bulge. In particular,
stellar population identification in the NIR can be aided greatly by the
use of narrow-band filters, to probe specific features and measure the
shapes of the SED (see Figure 1). Constructing detailed star formation
histories further helps us to better understand how star formation
proceeds in the extreme environment of the Galactic Center. However,
overcoming crowding means that this science generally must be assisted
by adaptive optics, else source confusion quickly limits the depth to
which we are able to probe. 

Finally, even although the bulge of the Galaxy is relatively metal-rich,
galaxy formation models suggest that the bulge may be home to the very
first stars that formed in the Universe (so-called Population III
stars). The first stars are believed to form in the highest density
peaks of the early Universe. At zero redshift, these ``peaks'' are to be
found in the centers of galaxies and galaxy clusters. Identification of
candidate stars requires not just high resolution imaging, but also
spectroscopy, ideally assisted with AO to deal with crowding. While
broad wavelength coverage at $R>5000$ is useful for candidate
identification, high resolution ($R>10000-20000$) is required to confirm
the identification of any primordial stars. 

\subsubsection{Milky Way globular clusters}

\begin{figure*}
  \begin{center}
%    \includegraphics[angle=270, width=12cm]{PMclean}\vspace{-0.5cm}
    \caption{Color-magnitude diagrams (CMDs) of globular cluster
   NGC188. The left panel shows the CMD of all stars within the field of
   view, whereas the right panel shows the CMD after removal of all
   stars with high proper motions (Galactic foreground stars). Note that
   the binary sequence running parallel to the main sequence is now
   observable in the cleaned CMD.} 
  \end{center}
\end{figure*}

Globular clusters are unique testbeds for stellar evolution and stellar
dynamics. Their formation and evolution is undoubtedly linked to the
formation of the Milky Way galaxy itself, albeit in ways that are still
not fully understood. Nearby globular clusters subtend relatively large
angles on the sky, and so wide fields of view are particularly
useful. They are also intrinsically crowded environments, and so well
sampled PSFs are essential. Roughly 160 globular clusters are known, and
the median extinction is around a magnitude. Further, in the central
regions of the Galaxy, we are likely missing some globular clusters;
numerous candidates are currently being identified in surveys such as
with UKIRT/WFCAM, and these require follow-up on larger facilities. Near
IR capabilities, ideally supported with adaptive optics to combat severe
crowding, are very useful to help understand the globular cluster
population of the Milky Way.  

As laboratories for the study of stellar populations, it is important
that analysis of the resolved stars in globular clusters accounts fully
for foreground contamination from intervening Galactic stars. Foreground
cleaning is possible with data with high astrometric accuracy:
comparison of images taken at two different epochs will show foreground
stars at slightly different positions relative to the more distant
globular cluster stars, due to their movement between epochs. Analysis
of the resulting CMD can often show features that were otherwise
difficult to identify, for example the binary sequence parallel to the
main sequence of the cluster (see Figure~2). 

With even higher astrometric precision, it is possible to start
constraining the internal dynamics of clusters through their internal
proper motions. Here the rms residuals when matching one epoch to the
other essentially measures the tangential velocity dispersion of the
cluster. Measuring this quantity as a function of radius is of
particular interest, since it is a matter of debate whether massive
black holes exist in the centers of clusters. If they do, they could
have a measurable effect on the dynamics of the stars at these radii,
potentially observable in the proper motions. 

\subsubsection{The nearest galaxies}


\begin{figure*}
  \begin{center}
%    \includegraphics[angle=0, width=7cm]{andi.ps}
%    \includegraphics[angle=0, width=7cm]{peg.ps}\vspace{-2cm}
%    \includegraphics[angle=0, width=6cm]{ic3104.ps}\vspace{-1cm}
    \caption{Structural studies of nearby (dwarf) galaxies. Top panels
   show two local group dwarf galaxies (Andromeda I and Pegasus) at
   distances of around 700kpc. The contours in the first panel and the
   grey-scale in the second panel show the distribution of individual
   stars in these galaxies, and act here as a proxy for surface
   brightness to very low light levels (fainter than
   30mags/sq.arcsec). The contours in the top right panel correspond to
   the HI distribution in Pegasus. Bottom panel: image of IC3104, at a
   distance of around 2.3Mpc. Here, contours tracing the diffuse light
   are overlaid on the $g-$band image. Bright foreground stars are also
   visible. A warp is present in this galaxy, even although it is very
   isolated. Star forming regions are visible. Observational studies of
   this galaxy with excellent IQ can help resolve the global stellar
   populations of this galaxy (in a similar way as for Local Group
   galaxies), examine its wide-field structure, and place future pointed
   observational studies in a global context.}
  \end{center}
\end{figure*}

Beyond the Milky Way lie the $\sim 70-80$ galaxies that are members of
the Local Group. Beyond the Local Group, there are roughly 20 or so
free-floating ``isolated'' galaxies, before we reach the next nearest
galaxy groups at around 3Mpc or so. All of these galaxies are within
reach of current ground based facilities that can resolve their
brightest stars in uncrowded regions. For the majority of galaxies
beyond the Local Group, this is generally limited to the far outer
regions of galaxies, since crowding otherwise prevents us from being
able to distinguish sources from one another in areas of high stellar
density. Further, the ability to distinguish stars from faint,
background galaxies prevents pushing these studies to faint stellar
populations (confusion with galaxies becomes dominant 0.5-1mag brighter
than the magnitude limit of the exposure). This type of resolved stellar
population study allows us to probe not only the ages and metallicities
of the stars, but also the spatial variation of these quantities and the
global structural properties of the galaxy. Figure 3 provides a few
examples of dwarf galaxies that are being studied in this way, in
addition to a more distant dwarf that is an idea target for the Subaru
telescope. 

Studies of nearby galaxies that are outside of the Local Group can offer
valuable insights into galaxy evolution by helping distinguish galactic
evolutionary processes that are internally-driven from externally-driven
environmental effects. Specifically, since dwarf galaxies are low mass,
they are very sensitive to various forms of internal feedback, for
example supernova-driven winds that can stop star formation and  
remove gas from the galaxy to inhibit future generations from
forming. However, examining these effects directly for the closest
dwarfs is complicated by the fact that the majority of nearby dwarfs are
satellite galaxies of either the Milky Way and M31. As satellites, dwarf
galaxies are easily perturbed by their massive hosts. This has a
significant effect on their evolution, for example by removing gas due
to ram pressure stripping or perhaps even by inducing star formation
after pericentric passages (for those dwarfs still with gas). Clearly,
these competing externally- and internally-driven effects obscure each
other, such that it is currently extremely difficult to disentangle
environmental effects from ``intrinsic'' processes. For dwarf galaxies
beyond the Local Group, however, that are not part of neighboring
groups, we can be sure (from timing arguments) that these galaxies have
not been significantly influenced by more massive galaxies. Examination
of their properties therefore can help establish what observed features
are due to processes that are intrinsic to the galaxy, and what effects
are primarily due to environment. 

Subaru is ideally placed to meet this science challenge. Most of the
dwarf galaxies have half-light radii that subtend a few arcmins, but
almost all are found to be very extended and the wide field view is
essential for structural and environmental studies. A stable, well
characterized PSF is essential, and the PSF must be well sampled in
order to be able to distinguish between stars and barely resolved
galaxies near the magnitude limit of the surveys. Small PSFs are
essential to successfully combat crowded fields. 

\subsubsection{Nearby galaxies in the era of the ELTs}


\begin{figure*}
  \begin{center}
%    \includegraphics[angle=0, width=7cm]{ngvs.ps}\vspace{-1cm}
    \caption{Structural studies of distant dwarf galaxies at the
   distance of the Virgo cluster, undertaken in excellent seeing
   conditions (IQ$\sim0.6''$) on CFHT as part of the Next Generation
   Virgo Cluster Survey (NGVS; see Ferrarese et al. 2012). While NGVS
   presents a uniform study of galaxies in the Virgo cluster, a
   comparative uniform wide field study of all galaxies between the
   Local Group and the Virgo cluster is completely lacking. These
   galaxies are the targets for next generation resolved stellar
   population studies with the ELTs. It is therefore essential that
   wide-field NIR photometric studies, ideally equipped with AO, are
   conducted in order to fully capitalize upon TMT resolved stellar
   population science.} 
  \end{center}
\end{figure*}

TMT will be able to obtain deep photometry and even spectroscopy of
individual stars at the distance of the Virgo cluster. Its field of view
is reasonably small, however, and it is not optimized to provide much
detailed information on the environment of the galaxies or their faint
outer regions. Consequently, studies of their global structure need to
be obtained elsewhere in order for forthcoming detailed studies of their
stellar content to be placed in the correct scientific context. {\it
Wide field NIR capabilities are essential to motivate and inform TMT
resolved stellar populations science}. 

At the distance of Virgo, 1\,arcsec corresponds to $\sim 80$pc. Thus,
even without GLAO, wide field surveys of galaxies out to the Virgo
cluster allow mapping of their structural and photometric properties
from the scale of tens of pc to many kpc. To do so, large fields of view
are desirable - at least $\sim 10\times10$ arcmins and preferably
larger. A well characterized, stable, uniform and well-sampled PSF is
essential in order to perform precision morphological modeling and
characterization of each galaxy.

As an example of what can be achieved for galaxies at the distance of
Virgo with good IQ, Figure~4 shows radial profiles derived from optical
ground based observations on CFHT (3.6m) in natural seeing, taken as
part of the Next Generation Virgo Cluster Survey (NGVS). Clearly,
observations with even better IQ and a well characterized PSF can probe
these galaxies at even smaller scales. Already, we are able to probe
their structures over orders of magnitude in scale. Multi-band studies
of these galaxies further probe stellar population variations, and set
the stage for science with the TMT, where the diffraction limited
capabilities allows resolved stellar population studies like those
currently underway for Local Group galaxies. 

Clearly, there already exists, through the NGVS, high quality
homogeneous data on the structures and colors of all galaxies in the
Virgo cluster. However, between the Local Group and Virgo cluster exists
a vast numbers of galaxy groups and ``free-floating'' galaxies, all of
which represent the targets for resolved stellar population studies in
the era of the ELTs. Most are lacking wide field data that probes outer
structures and environments, and certainly there is no homogeneous
analysis of the entire sample or significant subsample. Further, as a
NIR telescope, TMT science will benefit most from having complementary
NIR wide field data. Clearly, this is a significant and crucial role
that Subaru, equipped with a wide field NIR camera, is able to fulfill. 

\subsubsection{Science requirements and possible observing programs}


As is clear from the previous discussion, imaging is more important than
spectroscopy for this science, although studies of the Galactic Center
would certainly benefit from large, multiplexed, AO-assisted
spectrographs. For all science discussed herein, the following are
essential science requirements for an imaging capability: 

\begin{itemize}
\item Broad wavelength sensitivity/coverage (i.e., J, H, K minimum;
      I-band sensitivity desirable to give broad color baseline) 
\item Wide field of view. Minimum of $10 \times 10$\,arcmins. Wider
      field ($\sim$degree) highly desirable 
\item Well-sampled (critically sampled) PSF; stable and
      well-characterized PSF essential 
\item High throughput, to detect faint stars and low surface brightness
      features 
\item Standard broad band filters sufficient, but narrow-band filters
      desirable 
\end{itemize}


For the main imaging science, excellent IQ is required. This increases
sensitivity for the detection of faint objects,  overcomes crowding in
dense stellar environments and is better able to distinguish stars from
faint compact galaxies. A lot of this science is (marginally) possible
at around $IQ\sim0.6''$ (e.g., as exemplified by various studies with
Subaru SuprimeCam and CFHT MegaCam). Better IQ leads to rapidly
increasing science returns. GLAO is therefore an excellent option, and
in many respects the science studies described previously are ideal
science drivers for a GLAO-assisted NIR camera i.e., wide field,
confusion-limited, NIR science. However we note that the gains of GLAO
are lost if the PSF is not critically sampled. Since the delivered IQ
varies depending upon the natural seeing conditions, it is imperative
for this science that the PSF remains critically sampled even in the
best (20\%-ile) condition. 

A possible example of an instrument that could satisfy many of these
requirements is a wide-field camera that can operate at variable
plate-scales. For example, 3 different plate-scales of 0.05''/pix,
0.1''/pix, 0.2''/pix could be implemented, where the first two modes are
used when conditions are excellent or good. The decision as to which
mode is used is driven by the trade-off between good sampling of the PSF
versus required field of view (with the smaller plate scales
corresponding to reduced fields of view). The third mode is used when
natural seeing conditions are poor, such that the resulting IQ is
$\gsim 0.5''$, or when the largest field of view is required. Indeed,
such a mode could additionally be used {\it without} GLAO, perhaps
resulting in an even larger field of view.

Finally, it is worth noting that the development of an adaptive
secondary mirror (ASM) on Subaru is crucial for GLAO. However, it would
also provide a powerful foundation for future instrumentation
development (beyond 2020s), since an ASM makes future adaptive-optics
instruments or techniques (e.g., MOAO, high-contrast imaging, etc)
considerably easier to implement. If Subaru plans to continue NIR
instrumentation development for the long-term, then development of an
ASM would be a powerful investment for future instrumentation to
exploit. 

\subsection{Developing Subaru NIR instrumentation in context}

\subsubsection{Near-IR astronomy in the 2020s, Competition and synergies}

Construction of the Thirty Meter Telescope (TMT) will soon begin on
Mauna Kea. In the south, construction of the European Extremely Large
Telescope (E-ELT) moves ever closer. As correctly identified by Subaru
astronomers, it is essential that the investment in these facilities is
capitalized by ensuring the instruments and facilities exist to ``feed''
these giant telescopes and conduct complementary
observations. Developing a coherent and sustainable plan for telescope
operations in the era of the ELTs should be a major international
priority to ensure the best science can be conducted. Of course, this is
something that will require cooperation between international partners,
and this was nicely reflected in the attendance of the recent meeting on
GLAO science in Sapporo by both Japanese and Canadian astronomers. 

An obvious role for Subaru in the era of TMT is to provide the
wide-field NIR observations that complement the narrow field
observations of TMT. However, it is important to note a few other key
missions and instruments that could also impact the development of a new
NIR instrument on Subaru. 

\par\noindent
{\bf Space-based observatories}

Two major NIR space facilities will be launched towards the end of this
decade. The first, {\bf the James Webb Space Telescope (JWST)}, is a
general-purpose facility and is equipped with both imagers and
spectrographs. It will be unrivaled in its ability to peer into the deep
NIR universe, and obtain excellent spectra up to a resolution of
$R\sim3000$. The second, {\bf Euclid}, will be launched around 2020. It
will undertake a wide field imaging survey of the sky, surveying
15000-20000 square degrees and missing out only the region around the
Galactic plane. It will survey to a depth of Y=J=H=24 at S/N=5 for point
sources. It will have an encircled energy $EA50<0.3''$, and will use
$0.3''$/pixels (although it plans to dither in order to improve spatial
resolution). While the primary purpose of the survey is Dark Energy, the
wide-field NIR legacy imaging dataset that it will produce will be of
immense scientific interest. 

We note that there are other wide field NIR space observatories being
developed, in particular {\bf WISH} and {\bf WFIRST}. Both these
missions are at an earlier stage of development than JWST and Euclid. A
potentially interesting synergy exists with {\bf the Space Infrared
Telescope for Cosmology and Astrophysics (SPICA)}. Here, NIR observation
with Subaru would support the IR and far-IR observations of this
observatory, and could provide a powerful complement that could be
investigated further.  

For NIR observations, space will always be the preferred option, due to
the reduced sky and atmospheric seeing effects. However, it is important
to have complementary observations on the ground. In this respect, it is
vital that any new NIR imaging capability on Subaru... 

\begin{enumerate} 
\item ...has K-band sensitivity (to compliment Euclid)
\item ...is designed to concentrate on studies of the very faint
      Universe, magnitude $>24$ (to compliment Euclid)  
\item ...has a very wide field of view (to compliment JWST)
\end{enumerate}

\par\noindent
{\bf Ground-based instruments}

Aside from the TMT and E-ELT, there are several current and future
wide-field NIR instruments that are AO-assisted and that could impact
the design of a new Subaru capability. Most recent, {\bf Gemini/GeMS},
in the South, is a Multi-Conjugate Adaptive Optic (MCAO) system,
sensitive at J,H,K wavelengths and producing images in K with a 0.08''
FWHM over a 86'' field of view. It has recently been successfully
commissioned and appears to be working well. Since it is MCAO, its field
of view is smaller than achievable with GLAO, but with better
IQ. Gemini/GeMS is probably capable of much of the science discussed in
Section 1.2 (Milky Way globular clusters), since its field of view is
still large enough to observe most of the stars in a globular cluster,
although it would require many pointings to map the large scale
environment of these clusters (e.g., looking for tidal debris). 

Also in the southern hemisphere, VLT is currently converting one of its
Unit Telescopes into an Adaptive Optics Facility (AOF). As part of this
system, they are building a GLAO system,{\bf VLT/Graal}, for use with
Hawk-I, a $7.5'\times7.5'$ field of view imager. The performance of this
system is such that it is expected to become 1.5-2 times ``quicker''
than a seeing limited facility. According to the ESO webpages,  GLAO
will provide Hawk-I with a delivered IQ that is better than 0.5'' at
least 80\% of the time.  

While the Hawk-I camera has a smaller field of view than is possible on
Subaru, it can clearly address some of the science discussed in Section
1, particularly if it tiles observations to map a larger field. In this
respect, we note that it could be worthwhile to examine the case for
developing a wide field NIR instrument that is seeing-limited,
particularly if this means that efficiency can be gained elsewhere. For
example, the metric ``etendue'' is often used to quantify the
``efficiency'' of a camera. Etendue is the product of the telescope
effective area ($A_{eff}$) and the field of view of the camera
($\Omega$). GLAO leads to a higher value of $A_{eff}$ (since it is
quicker at detecting objects), but the value of $\Omega$ for a
GLAO-corrected field is usually considerably smaller than is possible
without adaptive optics. Therefore, {\it for science that is not
confusion-limited, it may be overall more efficient to build a
seeing-limited NIR camera with a very large field of view, rather than a
GLAO-assisted camera with a smaller  field of view}. A decision as to
which approach is better could be determined through examination of the
science, and by comparison of the maximum size of the seeing-limited
field of view to the size of the GLAO-corrected field of view. Clearly,
a seeing-limited camera would not greatly benefit the science discussed
in Section 1, since this science is wide field {\it and}
confusion-limited, and is ideal for a GLAO system. 

\subsubsection{Concluding thoughts}

A new wide-field, NIR capability on Subaru is an exciting and powerful
complement to the instrumentation that will be available on TMT, as well
as to numerous other facilities that will soon be available (e.g., JWST,
Euclid, SPICA). A GLAO-assisted imager would be a powerful tool for
wide-field, confusion-limited science studies, for which resolved
stellar populations and nearby galaxies could be very powerful science
drivers. Depending upon the other science that is envisioned, it could
also be very powerful to develop a seeing-limited, very-wide field
imager (assuming that the seeing-limited field of view is much larger
than the GLAO-corrected field of view). The science impact of either
option, while different, would both be powerful, and would be unique on
Mauna Kea. 

\vspace{2cm}
{\bf Acknowledgments:} We thank the organizers of the Subaru GLAO
science workshop in Sapporo for a very enjoyable and interesting
meeting. We thank Harvey Richer for providing several of the ideas
relating to GLAO observations of globular clusters. 

%\end{document}
