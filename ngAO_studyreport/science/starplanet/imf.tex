%%%%%%%%%%%%%%%%%%%%%%%%%%%%%%%%%%%%%%%%%%%%%%%%%%%%%%%%%%%%%%%%%%%%%%%%%%%%
%% Report format:
%%  - in English
%%  - in Tex(LaTex)
%%  - minimum 4 pages (maximum 10 pages?) per category
%%  - structure
%%      Introduction (summary of past researches, key questions)
%%      New windows to be opened with Subaru GLAO
%%      Proposed observations (obs mode, targets, required # of nights)
%%      Requirements for GLAO and instruments
%%$B!!(B    Synergy/Competitions with other facilities and/or on-going/planed projects
%%      (optional)
%%      Figures, Tables(optional)
%%      References
%%      Any pretty pictures for gravure? (optional)
%%%%%%%%%%%%%%%%%%%%%%%%%%%%%%%%%%%%%%%%%%%%%%%%%%%%%%%%%%%%%%%%%%%%%%%%%%%%

\def\thisdir{science/starplanet/}

%\documentclass[]{article}
%\usepackage{graphicx}
%\usepackage{amsmath,amssymb}
%\usepackage{natbib,aas_macros}
%\citestyle{aa}
%\bibliographystyle{apj}


%\oddsidemargin   -0.2cm
%\evensidemargin  -0.2cm
%\topmargin       -0.2cm
%\textwidth        16.5cm
%\textheight       22.0cm
%\parindent        11pt

%\def\gsim{\mathrel{\raise0.35ex\hbox{$\scriptstyle >$}\kern-0.6em % Greater/squiggles
%\lower0.40ex\hbox{{$\scriptstyle \sim$}}}}
%\def\lsim{\mathrel{\raise0.35ex\hbox{$\scriptstyle <$}\kern-0.6em % Less than/squiggles 
%\lower0.40ex\hbox{{$\scriptstyle \sim$}}}}

%\begin{document}

\section{The Origin of the Initial Mass Function\label{sec:IMF}}

\begin{center}
\noindent
{\bf M. Fukagawa, Y. Oasa, N. Narita}
\end{center}
\vspace{0.5cm}

\normalsize
%%      Introduction (summary of past researches, key questions)
\subsection{Introduction}
\subsubsection{The IMF: Universal or not?}
Understanding the origin of the initial mass function (IMF) has long
been an ultimate goal in the field of star formation, and it has a
crucial impact on study of galaxy evolution. Extensive efforts have been
performed to establish the IMF for a wide range of stellar mass and
cluster density, leading to a well-known functional form, so-called
Salpeter's law; $dN/d(\log m) \propto m^{\Gamma}$, $\Gamma = -1.35$
where $m$ is stellar mass \citep{sal55}. The Salpeter-like IMF appears
almost anywhere at least in the mass range of 0.4--10~$M_{\odot}$
although other types of representations have also been proposed
especially for stars less massive than 1~$M_{\odot}$
\citep{mil79,sca86,sca98,kro01}. The fact that the observed IMFs can be
reasonably well fitted by these functional forms yielded the idea of its
{\it universality}. However, the universality suggests that most, if not
all, stars form in the same way, which may sound unrealistic considering
star formation process in clusters with difference physical properties. 
% such as member density, metallicity, and UV radiation field.

In fact, there are compelling evidences of significant {\it variation}
of the IMF among   star-forming regions or within one cluster
\citep{oas06,hil97}. For instance, \cite{har08} found the flatter IMF
($\Gamma = -0.74$) for the massive Galactic young cluster, NGC 3603,
than the Salpeter-like one, confirming the top-heavy IMF in such
massive clusters and starbursts. At the same time, they identified the
different IMF slopes with the distance from the cluster center,
indicating that massive members are concentrated in the inner region
(Fig.~\ref{fig:imf}). It is worth noting that NGC 3603 is located at
7~kpc (with $A_V =$ 4--5) and harboring O- and B-type stars in a volume
of $<$1~pc$^3$, and the use of adaptive optics (AO) with VLT was
effective to study the IMF for a wide range of stellar mass at the
dense, central region of the cluster. 

Then, the question is, which physical/chemical process/property
determines the shape of the IMF. Is it turbulence, gravity, magnetic
field, metallicity, UV radiation, or interplay of those? To address
this, it is required to further develop observations to establish the
IMF in various environments by obtaining statistically meaningful
sample, where AO can play an important role.  

\subsubsection{How planetary-mass objects form?} 

The low-mass end of the IMF is intimately related to formation mechanism
of planetary-mass objects (PMOs) which is one of the most outstanding,
unsolved problems.  
Free-floating planets, gravitationally unbound to  host stars, have been
discovered in star-forming regions (e.g., \citealt{oas99}) as a natural
extension of the investigation of the stellar IMF. They could form in a
similar way to stars by fragmentation of molecular cloud cores, or in
circumstellar disks as bound planets but ejected by dynamical
instability. The bottom of the IMF can indicate the fragmentation
(opacity) limit, and hence it can reflect main building process for
PMOs. However, even in the deepest surveys, the detections were limited
to objects of a few to 5 Jupiter-masses, and the low-mass end has not
been established yet. On the other hand, the commonality of such
isolated PMOs was also recognized by microlensing, and the possible
change in the mass function at around one Jupiter-mass has been
suggested \citep{sum11}.  


There is one notable implication obtained by recent observations in
young clusters; PMOs found in such young regions are probably less
common than determined by microlensing in the field. In the clusters,
the mass spectrum smoothly extends into the planetary-mass regime down
to $\sim$6~M$_{\rm Jupiter}$ and the slope of the spectrum is not rising
\citep{sch12,pen12}. The planets less massive than 6~M$_{\rm Jupiter}$
might have the distinct mass spectrum. Alternatively, planets found in
microlensing might have experienced the different formation process such
as core accretion in circumstellar disks. It is thus definitely
important to go into the lower mass regime in future observations. It is
also useful to keep in mind when probing less massive members that the
mass of a young planet is usually estimated on the basis of the
evolutionary models which basically assume that planets form like
stars. Such models predict brighter objects than the calculations
supposing the formation through core accretion in  disks
(Fig.~\ref{fig:models}).  Higher sensitivity is the key for future
direct observations towards sub-Jupiter mass planets and to
appropriately interpret the bottom of the IMF. 


%%      New windows to be opened with Subaru GLAO
\subsection{New windows to be opened with Subaru GLAO}

Establishing the IMF under various star-forming environments requires
observations of distant clusters, which demands higher angular
resolution to separate each cluster member as well as higher
sensitivity. In massive star-forming regions, higher angular resolution
is also necessary to probe the cluster centers with being less
contaminated by the brightest stars. Surveys of star-forming regions
have been performed without AO so far since the primary requirement is a
wide field to count enough number of stars to construct the
IMF. Observations utilizing multi-conjugate AO has begun only recently,
but the field-of-view (FoV) is not quite large, $<2'$
\citep{pes13,bou09}.  
The combination of the wider field and AO will certainly enable
efficient observations of more clusters and their central regions.  

Observations basically consist of two steps; photometric detections of
candidate cluster members and spectroscopic follow-ups to accurately
estimate their temperature, surface gravity, and extinction, hence their
memberships and masses. Subaru/MOIRCS is already a powerful tool but it
is problematic that spectra are often overlapped with those of the
neighboring objects. Obtaining ``uncontaminated spectra'' is practically 
important for this science case, and GLAO can contribute to it.  

High sensitivity is important for detection and characterization of
PMOs. In addition, high observing efficiency provided by a large field
is essential to investigate the bottom of the IMF. The very deep
($\sim$2 hours on source), wide-field imaging with GLAO will provide the
detection limit $\sim$3~magnitudes better than the existing surveys of
young clusters. This improvement corresponds to pushing down the lowest
detectable mass from a few Jupiter-masses  to $\lsim$1~M$_{\rm Jupiter}$
for a cluster at 400~pc with $A_V < 1 $ and an age of $\sim$3~Myr, i.e.,
entering sub-Jupiter-mass regime.  On the other hand, observing distant
regions will improve the statistics of the lower-mass IMF, and for
instance, the limiting magnitudes of $J=25$ will enable detections of
massive, embedded PMOs at an age of 1~Myr in NGC 7538 ($d \sim 2.7$~kpc;
$A_V>15$). 


%%      Proposed observations (obs mode, targets, required # of nights)
\subsection{Proposed observations}

The study of the IMF will benefit from very deep, wide-field imaging and
MOS follow-up observations with GLAO. Approximately 30 star-forming
regions within 4~kpc can be potential targets. The required nights
depend on the cluster size, distance, and probably brightness contrast
between the brightest and faintest members. Nevertheless, crudely
assuming no mosaicking (given the planned FoV of $13.'6$ ) and on-source
integration of 2 hours, total $\sim$10 nights are required for the whole
cluster sample for the imaging. Spectroscopic follow-up may need one
night per cluster, and in this case, $\sim$40 nights are necessary. Note
that this is a very crude estimate ignoring the property of each
cluster. For instance, some nearby young clusters and associations are
sparse but good targets for finding less massive PMOs. More nights will
be required for such targets.  

%%      Requirements for GLAO and instruments
\subsection{Requirements for GLAO and instruments}
A wide-field NIR imager and multi-object spectrograph (MOS) is required
for this science case. Photometric estimate of the masses for the
detected sources is not impossible but  includes uncertainties, thus the
spectroscopic capability is needed to confirm the cluster memberships
and to obtain the better determination of their masses. If an integral
field spectrograph is available, disks/jet structures can also be
targeted. The wider FoV is better, but the FoV of $10'$--$14'$, which is
currently planned, would be good enough
(Fig~\ref{fig:spatial_dist}). This may be difficult to achieve, but
higher angular resolution, about $0.''1$ ($< 0.''2$), is quite valuable
to discriminate PMOs from background galaxies based on the morphology.  

It is worth noting that a wide-field NIR imager and MOS is very useful
for exoplanet science although GLAO itself is not critical due to its
relatively low strehl ratio. One of the significant progress in the
field of exoplant has been brought by the Kepler mission, and over 2500
planet candidates were discovered so far, including habitable planets,
Earth-sized or even smaller planets, and circumbinary ones
\citep{bor13,bor11}. However, Kepler's targets are relatively faint and
far, which makes radial-velocity follow-ups for all the candidates
difficult. Further study on atmospheric characterization is also hard to
perform. Thus, as a next step, future planet surveys will target nearby
bright stars to detect and characterize Earth-like or super-Earth
planets especially in the habitable zone, and one such space mission is
TESS (Transiting Exoplanet Surbey Satellite). TESS will employ transit
method for planet detection and currently planned to  be launched in
2017. The important, probably most exciting part of the follow-up of
TESS detections is atmospheric spectroscopy with large-aperture
telescopes. The NIR transmission spectrum is valuable  since NIR is
sensitive to molecules such as CH$_4$, CO, and CO$_2$ while the optical
one is useful to detect Rayleigh scattering \citep{nar13,ben12}
(Fig.~\ref{fig:trans_spec}). In addition, it is observationally a key to
simultaneously obtain spectra for target and reference stars, thus MOS
is the required capability for transmission spectroscopy
\citep{gib13}. It is also preferable to use a narrower slit to reduce
sky background while keeping the slit width to avoid light-loss, but
unfortunately, GLAO cannot contribute much to reduce the slit width due
to the moderate AO correction. Still, it should be emphasized that
wide-field NIR MOS is certainly  useful for characterizing exoplanetary
atmospheres and if a FoV of $\sim$10$'$ is achieved, it will become a
very unique capability among the 8-m class telescopes. 

\vspace{10mm}

\begin{figure}[htbp]
\centerline{
\includegraphics[height=50mm]{\thisdir fig1.eps}
}
\caption{IMFs for NGC 3603 obtained near ($left$) and far from ($right$)
 the cluster center with VLT \citep{har08}. The power-law index of the
 slope is $-0.74$ for the whole cluster, but the steepening with cluster
 radius was observed, indicating mass segregation in the inner
 region. The cluster center was observed with adaptive optics,
 successfully revealing the subsolar mass members in the core of this
 dense cluster. Diamonds and circles indicate the raw and the
 incompleteness corrected mass distributions, respectively. The best-fit
 power-law slopes are shown as a dotted and a dashed line.  The vertical
 dashed and dotted lines indicate the 50\% completeness limit within $r
 \sim 30''$ ($left$) and the detection limit in the outer field
 ($right$). }
\label{fig:imf}
\end{figure}

\begin{figure}
\centerline{
\includegraphics[height=45mm]{\thisdir fig2a.eps}
\hspace{10mm}
\includegraphics[height=45mm]{\thisdir fig2b.eps}
}
\caption{Models of cloudy atmospheres for a planet of 1--10~M$_{\rm
 Jupiter}$ at 400~pc \citep{spi12}. The solid lines correspond to the
 stellar-like formation (cloud fragmentation, disk instability) while
 the dotted ones represent the formation via core accretion in a
 circumstellar disk.}
\label{fig:models}
\end{figure}

\begin{figure}
\centerline{
\includegraphics[height=55mm]{\thisdir fig3a.eps}
\hspace{5mm}
\includegraphics[height=55mm]{\thisdir fig3b.eps}
}
\caption{$Left$: Color composite ($J, H, K_S$) image of S106 obtained
 with Subaru/CISCO \citep{oas06}. The FoV is about $5' \times
 5'$. $Right:$ Spatial distribution of sources detected in S106
 \citep{oas06}. The completeness limit lies in $18 < K' < 20.1$. The
 circles show YSO candidates and the crosses denote field stars. For
 instance, a source with $K'=19.3$ and $A_V=10$ at 1~Myr corresponds to
 6~M$_{\rm Jupiter}$ at the distance of S106 (600~pc).}
\label{fig:spatial_dist}
\end{figure}

\begin{figure}
\centerline{
\includegraphics[height=50mm]{\thisdir fig4a.eps}
\hspace{10mm}
\includegraphics[height=50mm]{\thisdir fig4b.eps}
}
\caption{$Left$: An example of a model transmission spectram of a
 super-Earth atmosphere \citep{ben12}. $Right$: Measured planet-to-star
 radius ratios for a transiting super-Earth GJ 1214b, compared with the
 two best-fit theoretical spectra for the water-rich atmosphere
 \citep{nar13}. A wide-field ($10'$) MOS will be a unique capability and
 very useful to investigate exoplanets' atmospheres.}
\label{fig:trans_spec}
\end{figure}

%\newpage
\bibliographystyle{apj}
\bibliography{\thisdir imf}

%\end{document}
