\def\thisdir{science/veryhighz/}


\section{Search for Galaxies at $z>7$ with Narrow-Band Imaging
\label{sec:nbf}}

\noindent
\begin{center}
%% Authors
{\bf Ikuru Iwata$^{1}$}\\
$^1$ Subaru Telescope, National Astronomical Observatory of Japan, 650
North Aohoku Place, Hilo, HI 96720, USA
\end{center}
\vspace{0.5cm}

\subsection{Introduction}

Subaru has been one of leading facilities pushing the frontier of
the distant universe. A unique capability of the prime focus camera
(Suprime-Cam) have enabled us to conduct wide-field survey which is
essentially important to find very rare objects such as luminous distant
galaxies. One of the efficient methods to find distant star-forming
galaxies is to detect Ly$\alpha$ emission using narrow-band filter (NBF) 
imaging. A strongly star-forming object with redshift 
$z = \lambda_{\mathrm NBF} / \lambda_{\mathrm Ly\alpha} -1$ could
appears to be bright comparared to those with adjacent broad-band
filters. Galaxies detected with this methods are called as 'Ly$\alpha$
emitters (or LAEs)'. The current most distant galaxy with a spectroscopic 
confirmation is an LAE at $7\approx 7.3$, which was discovered by
\citet{Shibuya2012} using Suprime-Cam with a narrow-band filter NB1006
(central $\lambda$ is 10052\AA).

Currently a new prime focus camera in optical wavelength,
Hyper Suprime-Cam (HSC), is under testing. HSC has more than seven times
wider field-of-view, and it is expected to enable us conducting deep
surveys much more efficiently than the current Suprime-Cam.
However, the wavelength of the redshifted Ly$\alpha$
T
Past researches.
Identify current issues.

\subsection{Surveys with ULTIMATE-SUBARU}

Methods.

What should be clarified with ULTIMATE-SUBARU.

\subsection{Proposed Observations}

Target objects: sample selection, number of objects, number of observing
fields, sky area.

Observing modes: imaging or spectroscopy.

Required observing time:

Special requirements for ULTIMATE-SUBARU other than baseline
specifications, if any.

\subsection{Synergy and Competitions}

\subsubsection{Synergy with TMT}

\subsubsection{Competitions with other facilities}

Instruments for 8--10m class telescopes.

ELT instruments.

Space-based projects.

\bibliographystyle{apj}
\bibliography{\thisdir nbf}
