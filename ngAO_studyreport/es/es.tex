\chapter{Executive Summary
\label{chap:exec_summary}}

\section{Background: Subaru Telescope Future Instrument Plans}

Subaru Telescope Science Advisory Committee (SAC) is an association of 
scientists in universities and institutions in Japan. As a
representative of Japanese optical-infrared astronomical community, they 
have made advisory comments and recommendations to Subaru Telescope,
NAOJ.
In the recommendation report published in March 2009, SAC listed
desirable future instruments of Subaru Telescope. There
are four candidates, namely,

\begin{enumerate}
\setlength{\itemsep}{-3pt}
\item Very Wide-field optical imager (which refers Hyper Suprime-Cam),
\item Wide-field multi-object spectrograph (which refers WFMOS that now
      turns into Prime Focus Spectrograph), 
\item Wide-field near-infrared (NIR) camera, and 
\item NIR integral-field spectrograph.
\end{enumerate}

The third and fourth candidates are NIR instruments. Unlike Hyper
Suprime-Cam and Prime Focus Spectrograph, there has been little activity
on the study of such future NIR instruments for Subaru Telescope.
In response to such recommendation from Japanese community, Subaru
Telescope initiated internal discussions on future instrumentations. 
During the course of such discussions, we recognized the significance of
wide-field adaptive optics system and wide-field NIR instrument
corresponding to such wide-field AO as a strong candidate of the Subaru
future instrument in NIR wavelengths.
Then we have formed the working group for the Subaru next-generation AO
system which consists of scientists not only in Subaru Telescope but
also some members in Japanese universities.

We had the first science workshop for Subaru next-generation AO in
September 2011 in Osaka, and based on science case discussions made
there and technical studies such as AO simulations and conceptual study
of optical design of NIR instrument, we published the study report in
August 2012 (in Japanese). 

This document is an English translation of the executive summary section
of the study report.

\medskip


\section{Subaru Telescope Next-Gen AO System}

There are some different ways to implement Wide-Field AO, and each of
them has its own characteristics (see
Section~\ref{sec:various_ao_systems}). Among them the working group
investigated extensively on two kinds of AO, namely,
{\bf Ground-Layer AO (GLAO)}, and {\bf Multi-Object AO (MOAO)}.

GLAO will achieve seeing improvement over the entire $\phi \gsim 10'$ field
of view (FoV) by correcting only for the disruption of images caused by the
ground layor of the Earth's atmosphere. Although the Strehl ratio with
GLAO should be much less than those by AO systems which achieve
diffraction limited images, you can achieve much wider FoV comparared to
the existing AO systems.

On the other hand, in MOAO, the wavefront errors not over the entire FoV
but toward multiple specific objects within a patrol area are
corrected. 

We made numerical simulations to evaluate performance of these systems
with Subaru Telescope. The details are described in 
Chapter~\ref{chap:simulation}.

\subsection{Ground Layer AO (GLAO)}

Our major findings from simulations for GLAO are as follows. 
Please refer Section~\ref{sec:glao_simulation} for details.

\begin{itemize}
 \setlength{\itemsep}{-3pt}
 \item Under a typical natural seeing condition (FWHM$\sim 0.4''$ in
       $K$-band), GLAO will be able to provide FWHM $\sim 0.2''$ in
       $K$-band. 
 \item Factor 1.5--2 gains are expected on ensquared energy for point
       sources, compared to observations under natural seeing.
 \item Approximately uniform wavefront correction over FoV diameter
       $\sim 20'$ will be achieved. Thus performance of GLAO would not
       be a primary factor in determining the FoV of the system;
       mechanical and optical limitations from the telescope and
       instruments would restrict available FoV.
 \item No significant performance degradation is expected with Tip-Tilt
       guide stars (TTGS) as faint as 18 mag., which is the limiting magnitude
       of the present AO188. Thus it is expected that the sky coverage
       of GLAO would be similar to that of AO188/LGS.
\end{itemize}

Since the fraction of the ground layer among atmospheric turbulence
components is thought to be relatively high in Mauna Kea, Subaru GLAO
could provide good performances. Additionally, we found that on-source
wavefront correction with the adaptive secondary mirror can be higher
than that achieved with the AO188.


\subsection{Multi-Object AO (MOAO)}

Details of simulations for MOAO are described in
Section~\ref{sec:moao_simulation}. Our major findings are as follows:

\begin{itemize}
 \setlength{\itemsep}{-3pt}
 \item If we use 6 GLSs, wavefront correction better than that of GLAO
       system will be achieved for those within a patrol area of a
       radius smaller than $\sim 3'$.

 \item In order to achieve good wavefront error corrections, TTGS should
       be reasonablly close to the target. Consequently, the sky
       coverage to be achieved with MOAO might be fairly similar to the
       FoV of the current AO188.
\end{itemize}

\medskip

Through these studies we have understood expected performances of GLAO 
and MOAO with the Subaru Telescope in reasonably realistic observing
conditions. MOAO will achieve high Strehl ratios for multiple targets,
but for the case with 8m-class telescopes their sky area where MOAO can
perform well will be limited; MOAO should be a strong system if it is
implemented for 30m-class telescopes.
GLAO can, on the other hand, provide seeing-improved wide-field images
which would not be easilly achieved with 30m-class telescopes. It has 
stronger synergy with 30m-class telescopes. We concluded that GLAO
is the primary candidate for the Subaru next-generation AO.

\bigskip

\cornersize{0.25}

\Ovalbox{\parbox{0.95\textwidth}{
\bf GLAO is our primary candidate as Subaru Telescope Next-generation AO
system. It will achieve FWHM $\sim0.2''$ @$K$-band over the entire
$>10'$ field of view, under a typical seeing condition.}}


\section{New Near-Infrared Instrument}

With GLAO we expecte almost uniform seeing improvement for FoV out to
$\sim 20'$. MOIRCS, the current near-IR imaging and multi-object
spectrograph for Subaru Telescope, has FoV of $4' \times 7'$. In order
to fully utilize the capability of GLAO, we definitely need to develop
new wide-field instruments. There are much more challenges in designing 
wide-field near-IR instrument comparared to designing instruments for
optical wavelength; fragile optical components with high throughput in
near-IR, need of cooling to suppress thermal emission, expensive
detectors, and so on. We started optical design studies of the new
wide-field near-IR instrument for the Cassegrain focus of Subaru
Telescope (Section~\ref{sec:inst_optics}).

We found that there is a possible design with a $13'$ diamter FoV for
the case with the same optical design parameters of the secondary
mirror as those of the current infrared secondary mirror. It was also 
found that if we chage the optical parameters of the primary and 
seconday mirrors, we will be able to achieve a $\sim 16'$ diameter
FoV.
There is no such wide-field near-IR instrument than the systems
consiedred here. With the seeing improvement achieved by GLAO, the
instrument should be very unique and competitive.

Although the current optical design is for imager and multi-object
spectrograph (using slit masks), 
from the science case discussions it has been pointed out that the 
Integral-Field Spectroscopy (IFS) is a key function. IFS will enable us
to resolve internal structures of extended objects, and it has become an
indispensable observing method for studying the galaxy evolution.
There are some IFS instruments which can be used with the assistance of
AO, such as VLT/SINFONI, Keck/OSIRIS, and Gemini/NIFS.
The spatial resolution achieved by GLAO should not be as high as those
by current single-conjugate AO systems, simultaneous IFS for multiple
objects in wide FoV can be very unique and strong capability.

We also recognize that GLAO will be able to improve image quality in
optical wavelength ($\lambda > 0.6 \mu$m). Possibilities of new
instruments in optical wavelength would be worth to be
explored. Moreover, the use of adaptive secondary mirror could reduce
the number of optics compared to classical AO systems, and thus thermal
background radiation from telescope and instruments should be
decreased. That would benefit observations in wavelength longer than
2$\mu$m. The adaptive secondary mirror (ASM) would also achieve very
high on-source strehl ratio. ASM / GLAO is not a single instrument but
can be regarded as significant telescope upgrade, and it will open up
various unique opportunities with Subaru Telescope.

\bigskip

\Ovalbox{\parbox{0.95\textwidth}{
\bf We have found an optical design of the Cassegrain instrument with
field-of-view of $13'$--$16'$. The GLAO-assisted wide-field NIR
instrument should have exciting capabilities over existing instruments.
}}


\section{Science with Wide-Field AO and New Instrument}

\subsection{Complete Sensus of the Galaxy Evolution with Large-Scale
  Near-IR Surveys}

Intensive studies of distant galaxies using 8--10m class telescopes, 
4m class survey telescopes as well as space telescopes in the past 15
years have revealed the outline of global history of galaxy formation
and evolution from the early stage of the universe to the present
epoch. We now know that the cosmic star formation rate density or the
average star formation rate for individual galaxies peaked around the
cosmic age of 2--5 billion years ($z\sim1-3$), and since then the global
star formation activity slowly is slowly declining. Stellar mass of
individual galaxies has continuted to grow, and morphologies of giant
galaxies such as spirals and ellipticals have emerged. At the same time,
it has been strongly suggested that super massive black holes which
reside in the center of galaxies have evolved in close connection with
star-formation activities. 
The big questions in the fieldd of galaxy evolution include:
\begin{itemize}
 \item what are the key parameters to drive the galaxy evolution among
       various phenomena that affect star formation activities in the
       galaxies?
 \item what detemines morphologies of the galaxies?
\end{itemize}
Since distant galaxies appear to be small, many of past researches have
been limited to observations of massive galaxies, and especially in many
cases internal structures of such distant galaxies have been neglected. 
Recent development of adaptive optics and sensitive integral field
spectrographs have enabled to resolve those distant galaxies. 

With {\it imaging observations}, we can obtain morphological information
such as size, radial profiles of light sources (stars and ionized gas),
assymetry, and color distribution. From the simulations of imaging
observations with GLAO (Sec.~\ref{sec:gal_sim_imaging}) we found that we
will be able to measure effective radii for less massive galaxies
compared to seeing-limited observations. Measurements of morphological
parameters for huge number of galaxies in various epochs and in wide 
range of mass will enable us to clarify the evolutionary paths of
stellar mass assembly, size, and morphology. We should also emphasize
that we can install new narrow-band filters which are designed to
capture important emission lines at specific redshifts to trace star
formation activities in the multiple galaxies within the target
field. Such addition of new filters and dispersion elements are one
significant advantage of ground-based facilities over the space-bourne
telescopes.

With {\it spectroscopic observations}, abundant physical information
such as star formation rate, amount of dust, metallicity, gas
kinematics, and outflow from galaxies into inter-galactic space. 
Especially, multi-IFS (or slit-scan observations using multi-slit masks)
is an effective way to collect `data cube' (spatial information and
spectral information). IFS studies of distant galaxies were made
primarily for those with the most active star formation at those
epochs, and because only one galaxy can be observed at a time, the
number of sample galaxies are limited (several tens at most). If we
consider the cost of telescope time, significant increase of the number
of sample galaxies through such single-object IFS would not be
expected. Survey using multi-IFS with a wide-field near-infrared 
instrument assisted by GLAO can be a very unique observing capability in
2020s. 
It is well known that the history of galaxy evolution strongly depends
on their environment. Systematics census of of (proto-)clusters of
galaxies including their outskirt is a key observation to understand
environmental effects, and GLAO + wide-field instrument is the best
instrument for such studies. 
The large survey of galaxies at $z\lsim 3$ with Subaru GLAO will produce
the first statistically robust (possibily integral-field) spectroscopic
database of distant galaxy populations.


\subsection{Discovery of the Most Distant Galaxies and Understanding the
  Cosmic Reionization with Narrow-band Imagign Surveys}

Deep observations in near-infrared are challenged by strong OH lines of
the Earth's atmosphere. However, we can suppress background noise for
imaging with narrow-band filters (NBF) which are designed to trace
photons with wavelengths between such strong OH lines. In
Section~\ref{sec:nbf} we discuss the possibility of searching Lyman
$\alpha$ emitting galaxies at $z>7$ with NBFs.
Researches based on the Subaru Telescope's unique capability of
wide-field imaging using the prime focus camera have achieved
discoveries of many of the most distant galaxies. For the redshift of
the current most distant galaxies, however, Lyman $\alpha$ emission is
redshifted to the very end of the wavelength coverage of the optical
instruments. We definitely need sensitive observations in the
near-infrared to push the frontier of the most distant galaxies
further and to understand the process of the cosmic reionization. 

Applications of NBF are not restricted to the search of distant Lyman
$\alpha$ emitting galaxies. Various studies such as systematic survey of
star-forming galaxies with H$\alpha$ emission should have a large
benefit of wide-field near-infrared instrument assisted by GLAO.


\subsection{Observing the Galactic Center}

\begin{itemize}
 \item Imaging and spectroscopy of globular clusters toward the Galactic
       center: kinematics of the Galactic bulge and dark matter
       distribution
 \item Nuclear star clusters as a key population to explore the
       co-evolution of the supermassive black hole and the Galactic bulge
 \item Wide-field astrometry of Hyper-velocity stars around the Galactic
       center
\end{itemize}


\section{Synergy with the Extremely Large Telescopes}

We aim at operation of Subaru GLAO around the end of 2010s and early
2020s. That is the epoch when Extremely Large Telescopes (ELTs) such as
the Thirty Meter Telescope are expected to start their first-light
observations. ELTs will have light collecting power and spatial
resolution substantially superior to the current 8--10m class
telescopes. Observations with ELTs will enable us to explore much
fainter targets, and investigate very details of internal structure of
various objects. On the other hand, wide-field (i.e., FoV larger than
$\gsim10'$) observation with ELTs are extremely challenging. Wide-field
capability of Subaru Telescope has been extended by the telescope
modification and installation of the new prime-focus camera Hyper
Suprime-Cam (1.5 deg. diameter), and the massive fiber-fed spectrograph 
Prime Focus Spectrograph (PFS)\footnote{The spectral coverage of PFS is
0.38$\mu$m -- 1.3$\mu$m.} is under development.
To implement observational capability which cannot be achieved by ELTs
and to execute observations complimentary to those by ELTs should be key
strategies for 8--10m class telescope in 2020s and later. 
A combination of Subaru GLAO and wide-field near-IR instruments is one
of the most significant projects which further develop the uniqueness
and advantage of Subaru Telescope and feed astronomical targets to ELTs
for detailed characterizations.

\bigskip

\Ovalbox{\parbox{0.95\textwidth}{
\bf Wide-field instruments for survey are key instruments in the era of
TMT and other ELTs. Subaru Telescope should work in cooperation with TMT
and be complimentary to the TMT's capabilities. The Development of a
wide-field near-infrared instrument is essentially important for Subaru
Telescope.}}


\section{Development Plan}

%$B3+H/7W2h$K$D$$$F$O(BChapter~\ref{chap:dev_plan}$B$K$F5-=R$7$F$$$k!#(B

\subsection{Development Organization and Funding}

\begin{itemize}
 \item Core group: next-generation AO study working group (Subaru
       Telescope AO team + scientists in Subaru and universities in
       Japan)
 \item Subaru Telescope staff + NAOJ (Mitaka) staff + international
       partnership
 \item Fund-raising from external competitive grants from the Japanese 
       government 
 \item NAOJ budget for Subaru Telescope modifications should be
       considered. 
 \item International partnerships are essentially important for both
       human resources and fudning.
\end{itemize}

\subsection{Development Schedule}

In order to maximize scientific outcomes, it is important to develop
the wide-field instrument with GLAO prior to the science observations of
TMTs, to execute survey observations, and to construct list of target
objects for detailed studies with TMT. We should construct a development
plan to start science observation by 2020.

